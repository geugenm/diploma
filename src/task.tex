\documentclass[14pt, a4paper]{extreport}

\usepackage[main=russian, english]{babel} 
\usepackage[a4paper, margin=2cm, right=2cm]{geometry}
\usepackage{tabularx}
\usepackage{ulem}
\usepackage{hyperref}
\usepackage{enumitem}
\usepackage{titling}
\usepackage{microtype}

\hyphenpenalty=0

\babelfont{rm}
[Path=../fonts/,
UprightFont=timesnrcyrmt-regular,
BoldFont=timesnrcyrmt-bold,
ItalicFont=timesnrcyrmt-italic,
BoldItalicFont=timesnrcyrmt-bold-italic,
Extension=.ttf
]{times_new_roman}

\pagestyle{empty}

\hypersetup{
    colorlinks=false,
}

\begin{document}

\begin{center}
    \vspace{0.5em}
    {{Форма задания по курсовой работе (курсовому проекту)}\\ 
    {Белорусский государственный университет}}
\end{center}

\noindent


\begin{tabularx}{\textwidth}{@{}lX@{}}
   {Факультет} & \uline{радиофизики и компьютерных технологий} \\ 
   {Кафедра} & \uline{физики и аэрокосмических технологий} \\ 
\end{tabularx}


\vspace{1em}
\begin{center}
    \textbf{ЗАДАНИЕ ПО КУРСОВОЙ РАБОТЕ (КУРСОВОМУ ПРОЕКТУ)}
\end{center}

\noindent 
\begin{enumerate}
\item {Студент} \uline{Глеба Евгений Михайлович}
\item {Тема} \uline{ОЦЕНКА ВЛИЯНИЯ СОЛНЕЧНОЙ АКТИВНОСТИ НА РА\-БО\-ТУ БОРТОВЫХ СИСТЕМ НИЗКООРБИТАЛЬНЫХ СПУ\-ТНИ\-КОВ С ИСПОЛЬЗОВАНИЕМ АЛГОРИТМА МАШИННОГО ОБУ\-ЧЕ\-НИЯ XGBOOST И ПОСТРОЕНИЯ ГРАФА СВЯЗНОСТИ} 
\item {Срок представления курсовой работы (курсового проекта) к защите} \underline{27 декабря 2024 г.} 
\item {Исходные данные к курсовой работе (курсовому проекту)} (при необходимости)
    \begin{enumerate}[label=\arabic{enumi}.\arabic*]
        \item \uline{B. K. Ray и R. S. Tsay. «Bayesian methods for changepoint detection in long-range dependent processes». В: Journal of Time Series Analysis 23(6) (2002)} 
        \item \uline{J. C. Green, J. Likar и Y. Shprits. «Impact of space weather on the satellite
        industry». В: Advancing Earth and space science 15 (2017)} 
        \item \uline{Zachary Luppen, Dae Young Lee и Kristin Y. Rozier. «Introducing formal methodologies to monitor small aerospace system telemetry». В: 2021 IEEE International Symposium on Software Reliability Engi\-nee\-ring Workshops (ISSREW), 15--22} 
        \item \uline{R. Boumghar и др. «Enhanced awareness in space operations using multipurpose dynamic network analysis». В: Space Operations: Inspi\-ring Humankind’s Future. Springer International Publishing, 2018} 
    \end{enumerate}
\item {Содержание (структура) курсовой работы (курсового проекта):}
    \begin{enumerate}[label=\arabic{enumi}.\arabic*]
        \item \uline{Постановка цели и задач исследования (введение)} 
        \item \uline{SatNOGS: Глобальная сеть наземных спутниковых станций с откры\-тым исходным кодом} 
        \item \uline{Глубокая модификация Polaris ML: polaris 2.0} 
        \item \uline{Анализ графов спутников полученных при помощи polaris 2.0} 
    \end{enumerate}
\end{enumerate}

\vspace{2em}

\noindent
\begin{tabularx}{\textwidth}{@{}|X|X@{}|}
    \hline
    \multicolumn{1}{|c|}{\textbf{Содержание задания}} & \multicolumn{1}{c|}{\textbf{Сроки выполнения}} \\
    \hline
    Разработка системы парсинга и логгирования данных с SatNOGS Dashboard & 01.09.2024 -- 20.10.2024 \\
    \hline
    Создание нового backend'а polaris 2.0. Оптимизация построения графов  & 20.10.2024 -- 15.11.2024 \\
    \hline
    Анализ и выбор спутников с наиболее репрезентативными данными & 15.11.2024 -- 20.12.2024 \\
    \hline
\end{tabularx}

\vspace{2em}

\noindent
\begin{tabularx}{\textwidth}{@{}l@{\hspace{0.5cm}}l@{\hspace{1cm}}X@{}}
  {Руководитель курсовой}  \\
  {работы (курсового} &  & \\ 
  {проекта)} & \hrulefill & \hrulefill \\
                                & \centering (подпись, дата) & \centering (инициалы, фамилия)
\end{tabularx}

\vspace{3em}

\noindent
\begin{tabularx}{\textwidth}{@{}l@{\hspace{5pt}}X@{}}
    Задание принял к исполнению & \hrulefill \\
    & \centering (подпись, дата)
\end{tabularx}

\vspace{1em}

\noindent {Проинформирован о недопустимости привлечения третьих лиц к выполнению курсовой работы (курсового проекта), плагиата, фальсификации или подлога материалов.} 

\vspace{2em}

\begin{tabularx}{\textwidth}{XX}
  \hrulefill & \hrulefill \\
  \centering (подпись) & \centering (инициалы, фамилия обучающегося) 
\end{tabularx}

\vspace{1em}

\noindent \textbf{ПРИМЕЧАНИЕ.} Допускается дополнять или исключать пункты в бланке задания. 

\end{document}