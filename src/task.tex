\begin{center}
\textbf{\large Белорусский государственный университет}
\end{center}

\noindent Кафедра \remainingfield{1.8cm}{физики и аэрокосмических технологий}

\vspace{0.2cm}

\begin{flushright}
\begin{tabular}{@{}l@{}}
УТВЕРЖДАЮ \\
Заведующий (начальник) \\
кафедрой \\[0.2cm]
\begin{tabular}{@{}l@{\hspace{1.8cm}}l@{}}
\ufield{4.2cm}{} & \ufield{5.5cm}{Саечников В.А.} \\
\makebox[4.2cm][c]{\scriptsize (подпись)} & \makebox[5.5cm][c]{\scriptsize (фамилия, инициалы)}
\end{tabular} \\[0.8cm]
\ufield{3cm}{} 20\ufield{1cm}{} г.
\end{tabular}
\end{flushright}

\begin{center}
\textbf{\large ЗАДАНИЕ} \\
\textbf{\large на дипломную работу (дипломный проект)}
\end{center}

\noindent Обучающемуся \remainingfield{3.4cm}{Глеба Евгению Михайловичу} \\
\hspace*{3.2cm}\makebox[\textwidth-3.4cm][c]{\scriptsize (фамилия, собственное имя, отчество (если таковое имеется))}

\noindent Курс \ufield{1.8cm}{4} \hspace{1.5cm} Учебная группа \remainingfield{8cm}{8АРИСТ}

\noindent Специальность \remainingfield{3.3cm}{1-31 04 04 "Аэрокосмические радиоэлектронные и информа-} \\
\fullfield{ционные системы и технологии"} \\

\noindent Тема дипломной работы (дипломного проекта) \remainingfield{10cm}{Оценка влияния солнечной} \\
\fullfield{активности на работу бортовых систем низкоорбитальных спутников с} \\
\fullfield{использованием алгоритма машинного обучения XGBoost и построения } \\
\fullfield{графа связности} \\
\hspace*{5cm}\makebox[\textwidth-9cm][c]{\scriptsize (наименование темы)} \\
\noindent Утверждена приказом ректора БГУ от \ufield{3.5cm}{25.11.2024} № \remainingfield{12.5cm}{1358-ПС} \\
\noindent Исходные данные к дипломной работе (дипломному проекту) \remainingfield{13cm}{Данные косми-} \\
\fullfield{ческой погоды, телеметрия спутников SatNOGS, алгоритмы машинного} \\
\fullfield{обучения XGBoost, методы анализа временных рядов и построения графов}

\vspace{0.1cm}

\noindent Перечень подлежащих разработке вопросов или краткое содержание расчет-\\
но-пояснительной записки \remainingfield{5.9cm}{Анализ влияния солнечной активности на } \\
\fullfield{функционирование спутниковых систем, разработка и обучение модели} \\
\fullfield{XGBoost, построение графа связности, валидация результатов прогнозиро-}
\fullfield{вания}

\vspace{0.1cm}

\noindent Перечень графического материала (с точным указанием обязательных чер-\\
тежей и графиков) \remainingfield{4cm}{} \\
\fullfield{} \\
\fullfield{}

\vspace{0.1cm}

\noindent Консультанты по дипломной работе (дипломному проекту) \\
(с указанием разделов, по которым они консультируют) \remainingfield{12cm}{} \\
\fullfield{} \\
\fullfield{}

\vspace{0.1cm}

\noindent Примерный календарный график выполнения дипломной работы \\
(дипломного проекта) \remainingfield{4.8cm}{февраль 2025: анализ литературы по космической} \\
\fullfield{погоде и спутниковым системам;} \\
\fullfield{март 2025: сбор и предобработка данных SatNOGS;} \\
\fullfield{апрель 2025: разработка модели XGBoost и графа связности;} \\
\fullfield{май 2025: тестирование, валидация и оформление результатов работы}

\vspace{0.2cm}

\noindent Дата выдачи задания \remainingfield{12cm}{5 февраля 2025 г.}

\vspace{0.1cm}

\noindent Срок сдачи законченной дипломной работы \\
(дипломного проекта)
\hspace*{5cm} \remainingfield{12cm}{} \\


\vspace{0.1cm}

\noindent Руководитель дипломной работы \\
(дипломного проекта) \\[0.1cm]
\begin{tabular}{@{}l@{\hspace{2cm}}l@{}}
\ufield{4.5cm}{} & \ufield{6cm}{Баранова В.С.} \\
\makebox[4.5cm][c]{\scriptsize (подпись)} & \makebox[6cm][c]{\scriptsize (инициалы, фамилия)}
\end{tabular}

\vspace{0.2cm}

\noindent Подпись обучающегося  \remainingfield{12cm}{}

\vspace{0.2cm}

\noindent Дата \ufield{2.5cm}{} 20\ufield{1cm}{} г.

\vspace{0.2cm}

\noindent Проинформирован о недопустимости привлечения третьих лиц к выполнению \\
дипломной работы (дипломного проекта), плагиата, фальсификации или подлога материалов.

\vspace{0.2cm}

\begin{tabular}{@{}l@{\hspace{2.5cm}}l@{}}
\ufield{4.5cm}{} & \ufield{6cm}{Е. М. Глеба} \\
\makebox[4.5cm][c]{\scriptsize (подпись)} & \makebox[6cm][c]{\scriptsize (инициалы, фамилия обучающегося)}
\end{tabular}