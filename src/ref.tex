\chapter*{РЕФЕРАТ}

\textbf{Структура и объем дипломной работы}: дипломная работа содержит \pageref{LastPage}~страниц, \totalfigures~рисунков, \totaltables~таблиц, 1~приложение, 43~источника.

\textbf{Ключевые слова}: КОСМИЧЕСКАЯ ПОГОДА, СОЛНЕЧНАЯ АКТИВНОСТЬ, МАШИННОЕ ОБУЧЕНИЕ, XGBOOST, СПУТНИКОВЫЕ СИСТЕМЫ, ТЕЛЕМЕТРИЯ, ГРАФ СВЯЗНОСТИ, КРОСС-КОРРЕЛЯЦИЯ, НИЗКООРБИТАЛЬНЫЕ СПУТНИКИ, SATNOGS, POLARIS ML, ПАРАЛЛЕЛЬНЫЕ ВЫЧИСЛЕНИЯ.

\textbf{Объект исследования}: низкоорбитальные спутники и их бортовые системы в условиях воздействия космической погоды.

\textbf{Предмет исследования}: взаимосвязи между параметрами солнечной активности и электротехническими характеристиками космических аппаратов.

\textbf{Цель работы}: разработка методологии анализа влияния космической погоды на функционирование спутниковых систем с использованием алгоритмов машинного обучения и построения графов связности.

\textbf{Методы исследования}: модифицированный алгоритм XGBoost, анализ графов кросс-корреляций, параллельные вычисления, статистический анализ временных рядов, веб-скрапинг телеметрических данных из сети SatNOGS.

\textbf{Полученные результаты и их новизна}: создана усовершенствованная платформа Polaris ML с модифицированным алгоритмом XGBoost, обеспечивающая точность F1-Score 0.94 при 25-кратном ускорении обработки данных. Впервые проведен комплексный анализ графов кросс-корреляций для спутников GRIFEX, ENSO, Veronika, LASARsat и INSPIRESat-1, выявивший нелинейные зависимости между солнечной активностью и надежностью бортовых систем. Разработан математически обоснованный подход к параллельному вычислению кросс-корреляций между временными рядами.

\textbf{Достоверность результатов}: подтверждается использованием данных глобальной сети SatNOGS, применением проверенных статистических методов, валидацией результатов на множественных спутниковых миссиях и интеграцией MLflow для воспроизводимости экспериментов.

\textbf{Практическая значимость}: разработанная система позволяет существенно сократить время обработки телеметрии малых космических аппаратов и выявлять критические взаимосвязи между космической погодой и функционированием бортовых систем, что способствует повышению надежности спутниковых миссий.

\textbf{Рекомендации по внедрению}: интеграция разработанных методов в существующие системы управления спутниками для принятия обоснованных решений по изменению режимов работы бортовой аппаратуры в периоды повышенной солнечной активности.

\textbf{Область применения}: космическая отрасль, системы мониторинга и управления спутниками, научные исследования космической погоды.
