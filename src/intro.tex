\nonPrefixChapter{Введение}

Космическая погода представляет собой сложный набор явлений, происходящих в околоземном пространстве, которые оказывают значительное влияние на функционирование низкоорбитальных спутников. К основным факторам космической погоды относятся солнечные вспышки, корональные выбросы массы и геомагнитные бури. Эти явления могут вызывать аномалии в телеметрии, сбои в работе бортовых систем и даже полную потерю спутника.

Проблема влияния космической погоды на функционирование спутниковых систем приобретает особую актуальность в контексте экспоненциального роста количества космических аппаратов на околоземной орбите. 

Физические механизмы воздействия космической погоды на электронные компоненты спутников многообразны и включают радиационное повреждение полупроводниковых элементов, индукцию паразитных токов в электрических цепях, деградацию солнечных батарей и изменение параметров орбиты вследствие расширения верхних слоев атмосферы. Особую опасность представляют высокоэнергетические заряженные частицы, способные вызывать одиночные сбои (Single Event Upsets, SEU) в микроэлектронике, что может привести к критическим ошибкам в работе бортового программного обеспечения.

В последнее время мы значительно модернизировали систему Polaris, внедряя более широкий анализ данных от множества спутников и выявляя критические системы, наиболее подверженные воздействию космической погоды. Архитектура системы была переработана с учетом современных требований к обработке больших данных, что позволило интегрировать информацию с более чем 150 различных датчиков на каждом из анализируемых космических аппаратов. Это дало возможность создать беспрецедентно детальную картину функционирования бортовых систем в условиях различных космических явлений.

Одним из ключевых направлений нашей работы стала разработка специализированных интерфейсов для фильтрации параметров солнечной погоды. Эти инструменты позволяют операторам спутниковых систем в режиме реального времени отслеживать потенциально опасные космические явления и прогнозировать их влияние на конкретные компоненты бортовой электроники. Реализованные алгоритмы фильтрации основаны на методах спектрального анализа и вейвлет-преобразований, что обеспечивает высокую точность выделения значимых сигналов на фоне шумов различной природы.

Современные автоматизированные системы мониторинга состояния телеметрии, использующие различные модели машинного обучения, становятся все более актуальными в условиях растущей сложности космических аппаратов и увеличения объема передаваемых данных. Эти системы позволяют оценивать работоспособность бортовых систем как в процессе наземных испытаний, так и в режиме полетной диагностики, что критически важно для обеспечения надежности космической инфраструктуры.

Спутник представляет собой сложную электрическую систему с высокой степенью интеграции компонентов, где сбои на одном узле могут вызвать каскадную реакцию неисправностей, затрагивающую множество подсистем. Причины сбоев могут быть как внутренними, связанными с естественными неполадками компонентов и программного обеспечения, так и внешними, вызванными воздействием факторов космической погоды. Дифференциация этих причин представляет собой нетривиальную задачу, требующую применения комплексного подхода, включающего анализ временных рядов, методы машинного обучения и экспертные системы.

В данной работе мы продолжаем оценивать взаимосвязь между солнечной активностью и работой бортовой электроники на основе данных телеметрии космических аппаратов, находящихся на различных орбитах и выполняющих разнообразные миссии. Наш подход основан на гипотезе о существовании статистически значимых корреляций между определенными параметрами космической погоды и аномалиями в функционировании электронных компонентов. Для проверки этой гипотезы мы разработали методологию, сочетающую классические статистические тесты с современными методами глубокого обучения, адаптированными для работы с многомерными временными рядами.

Мы также занимаемся разработкой и оптимизацией открытых решений для анализа больших данных, что позволяет нам углубленно исследовать влияние космической погоды на спутниковые системы. Особое внимание уделяется созданию масштабируемой инфраструктуры для параллельной обработки телеметрической информации, что критически важно при работе с данными, поступающими в режиме реального времени от множества космических аппаратов. Реализованные нами алгоритмы распределенных вычислений позволили сократить время обработки телеметрических данных в 12 раз по сравнению с традиционными подходами.

Научная новизна нашего исследования заключается в разработке комплексной методологии анализа влияния космической погоды на функционирование спутниковых систем, учитывающей как прямые, так и отложенные эффекты. В отличие от предыдущих работ, фокусировавшихся преимущественно на краткосрочных последствиях солнечных вспышек, мы исследуем долговременные эффекты накопления радиационных повреждений и их влияние на деградацию характеристик электронных компонентов. Это позволяет более точно прогнозировать срок службы спутников и планировать профилактические мероприятия.

Практическая значимость исследования определяется возможностью использования его результатов для повышения надежности космических аппаратов и снижения рисков их выхода из строя вследствие воздействия неблагоприятных факторов космической погоды. Разработанные нами методы и инструменты могут быть интегрированы в существующие системы управления спутниками, что позволит операторам принимать обоснованные решения по изменению режимов работы бортовой аппаратуры в периоды повышенной солнечной активности.
