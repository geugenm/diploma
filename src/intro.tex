\nonPrefixChapter{Введение}

Космическая погода представляет собой сложный набор явлений, происходящих в
околоземном пространстве, которые оказывают значительное влияние на
функционирование низкоорбитальных спутников. К основным факторам космической
погоды относятся солнечные вспышки, корональные выбросы и геомагнитные бури.
Эти явления могут вызывать аномалии в телеметрии, сбои в работе бортовых систем
и даже полную потерю спутника. В последнее время мы значительно модернизируем
систему Polaris, внедряя более широкий анализ данных от множества спутников и
выявляя критические системы. В рамках этой работы мы начали использовать
логирование обучений моделей с помощью MLflow и добавили графики для
визуализации результатов моделей. Также мы разрабатываем интерфейсы для
фильтрации параметров солнечной погоды, что позволяет более эффективно
отслеживать и анализировать влияние космической активности на работу бортовой
электроники. Современные автоматизированные системы мониторинга состояния
телеметрии, использующие различные модели машинного обучения, становятся все
более актуальными. Эти системы позволяют оценивать работоспособность бортовых
систем как в процессе испытаний, так и в режиме полетной диагностики. Спутник
представляет собой сложную электрическую систему, где сбои на одном узле могут
вызвать цепную реакцию неисправностей. Причины сбоев могут быть как
внутренними, связанными с естественными неполадками компонентов, так и
внешними, вызванными воздействием космической погоды. В данной работе мы
продолжаем оценивать взаимосвязь между солнечной активностью и работой бортовой
электроники на основе данных телеметрии космических аппаратов. Мы также
занимаемся разработкой и оптимизацией открытых решений для анализа больших
данных, что позволяет нам углубленно исследовать влияние космической погоды на
спутниковые системы.
