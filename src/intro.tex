\chapter*{Введение}

\textbf{1. Описание проблемы}

Проблема влияния космической погоды на функционирование спутниковых систем 
приобретает критическую важность в условиях экспоненциального роста количества 
космических аппаратов на околоземной орбите[2][6]. Физические механизмы воздействия 
космической погоды включают радиационное повреждение полупроводниковых элементов, 
индукцию паразитных токов в электрических цепях, деградацию солнечных батарей и 
изменение параметров орбиты вследствие расширения верхних слоев атмосферы[2][6]. 
Высокоэнергетические заряженные частицы способны вызывать одиночные сбои (Single 
Event Upsets, SEU) в микроэлектронике, что приводит к критическим ошибкам в работе 
бортового программного обеспечения.

Современные спутниковые системы представляют собой сложные электрические комплексы 
с высокой степенью интеграции компонентов, где сбои в одном узле могут инициировать 
каскадные реакции неисправностей, затрагивающие множество подсистем. Дифференциация 
внутренних причин сбоев (естественные неполадки компонентов и программного обеспечения) 
от внешних воздействий факторов космической погоды представляет нетривиальную 
научно-техническую задачу[6].

\textbf{2. Существующие подходы к решению проблемы}

Современные методы прогнозирования космической погоды основываются на двух основных 
подходах: практическом, направленном на предсказание и смягчение негативных проявлений, 
и фундаментальном, описывающем взаимодействие солнечного ветра с магнитосферой Земли[6]. 
Существующие системы мониторинга используют данные космических аппаратов ACE, SOHO и 
"Винд" для отработки алгоритмов прогнозирования[6].

Применяются автоматизированные системы мониторинга состояния телеметрии, использующие 
модели машинного обучения для оценки работоспособности бортовых систем как в процессе 
наземных испытаний, так и в режиме полетной диагностики. Однако существующие подходы 
фокусируются преимущественно на краткосрочных последствиях солнечных вспышек, не 
учитывая долговременные эффекты накопления радиационных повреждений[6].

\textbf{3. Предлагаемое решение и цель работы}

В данной работе разработана комплексная методология анализа влияния космической 
погоды на функционирование спутниковых систем, учитывающая как прямые, так и 
отложенные эффекты. Основной целью исследования является создание инструментов 
для повышения надежности космических аппаратов и снижения рисков их выхода из 
строя вследствие воздействия неблагоприятных факторов космической погоды.

Ключевыми компонентами разработанного решения являются:
\begin{itemize}
\item Специализированные интерфейсы для фильтрации параметров солнечной погоды, 
основанные на методах спектрального анализа и вейвлет-преобразований
\item Масштабируемая система анализа больших данных с распределенными алгоритмами 
обработки телеметрической информации
\item Методология, сочетающая классические статистические тесты с современными 
методами глубокого обучения для работы с многомерными временными рядами
\end{itemize}

Научная новизна исследования заключается в комплексном подходе к анализу 
долговременных эффектов накопления радиационных повреждений и их влияния на 
деградацию характеристик электронных компонентов. Практическая значимость 
определяется возможностью интеграции разработанных методов и инструментов в 
существующие системы управления спутниками для принятия обоснованных решений 
по изменению режимов работы бортовой аппаратуры в периоды повышенной солнечной 
активности.
