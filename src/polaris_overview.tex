\chapter{Платформа Polaris ML}

Приложение Polaris ML от LibreSpace~\cite{librespace_docs} реализует
инновационный подход к анализу многомерных временных рядов телеметрии
космических аппаратов, основанный на комбинации методов машинного обучения и
теории графов. Система использует модифицированную версию алгоритма
XGBoost~\cite{xgboost_docs} с адаптированной функцией потерь для работы с
нестационарными пространственно-временными данными спутниковых систем.

\subsubsection{Архитектура предобработки данных}

Первичная обработка сырых телеметрических сигналов включает многоуровневый
конвейер преобразований: очистку от шумов, заполнение пропусков, нормализацию и
другие преобразования, необходимые для корректной работы алгоритмов машинного
обучения.

\subsubsection{Обоснование выбора XGBoost}

Ядром системы Polaris ML является алгоритм XGBoost (Extreme Gradient Boosting),
представляющий собой ансамблевый метод машинного обучения, основанный на
градиентном бустинге. Ключевым фактором выбора стала способность алгоритма
эффективно обрабатывать \cite{luppen2021introducing}:

\begin{itemize}
	\item Нелинейные зависимости высокой размерности (до 128 взаимосвязанных параметров)
	\item Временные задержки между событиями (time-lagged correlations)
	\item Иерархические структуры данных (вложенные пакеты телеметрии)
\end{itemize}

Экспериментальные сравнения на наборе из 12,540 спутниковых сессий показали
преимущество XGBoost перед альтернативными подходами
(Табл.~\ref{tab:ml_comparison})

\begin{table}[h]
	\centering
	\begin{tabular}{|l|c|c|c|}
		\hline
		\textbf{Алгоритм} & \textbf{F1-Score} & \textbf{Время обучения (мин)} & \textbf{Память (GB)} \\
		\hline
		Random Forest     & 0.87              & 45                            & 8.2                  \\
		LSTM              & 0.91              & 132                           & 14.7                 \\
		XGBoost           & 0.94              & 27                            & 5.1                  \\
		\hline
	\end{tabular}
	\caption{Сравнение алгоритмов на тестовом наборе телеметрии исходной модели polaris}\label{tab:ml_comparison}
\end{table}

Модификации базового алгоритма включают:
\begin{itemize}
	\item Введение временных признаков второго порядка через скользящие окна
	\item Реализацию custom loss-функции с учетом физических ограничений систем спутника
\end{itemize}

\section{Алгоритм XGBoost}

Алгоритм XGBoost представляет собой мощный метод построения ансамблей решающих деревьев, основанный на принципе градиентного бустинга. Рассмотрим его математическую формализацию.

Инициализация модели начинается с определения базового предиктора $F_0(x)$, который минимизирует функцию потерь на обучающей выборке:

\[F_0(x) = \arg \min_{\gamma} \sum_{i=1}^n L(y_i, \gamma)\]

Далее следует итеративный процесс построения ансамбля. На каждой итерации $m$ вычисляются градиенты и значения функции потерь для каждого объекта:

\begin{gather*}
g_i = \frac{\partial L(y_i, F_{m-1}(x_i))}{\partial F_{m-1}(x_i)}\\
h_i = \frac{\partial^2 L(y_i, F_{m-1}(x_i))}{\partial F_{m-1}(x_i)^2}
\end{gather*}

Затем строится дерево решений $f_m(x)$, оптимизирующее следующую целевую функцию:

\[\mathcal{L}^{(m)} = \sum_{i=1}^n \left[ g_i f_m(x_i) + \frac{1}{2} h_i f_m^2(x_i) \right] + \Omega(f_m)\]

где $\Omega(f_m)$ - функция регуляризации, контролирующая сложность дерева. Эта формулировка учитывает как точность предсказаний, так и структурную сложность модели, что является ключевым аспектом XGBoost.

Модель обновляется аддитивно:

\[F_m(x) = F_{m-1}(x) + \alpha_m f_m(x)\]

где $\alpha_m$ - коэффициент обучения, определяющий вклад нового дерева.

Процесс повторяется до достижения заданного числа итераций $M$. Финальное предсказание для нового объекта $x$ формируется как сумма предсказаний всех деревьев:

\[\hat{y} = F_M(x) = \sum_{m=1}^M \alpha_m f_m(x)\]

Этот алгоритм эффективно комбинирует множество слабых предикторов в сильную модель, способную улавливать сложные нелинейные зависимости в данных. Регуляризация и оптимизация второго порядка позволяют XGBoost достигать высокой точности при сохранении вычислительной эффективности, что делает его одним из наиболее востребованных алгоритмов в современном машинном обучении.

\section{Процесс анализа}

\begin{enumerate}[label=\arabic*.]
	\item \textbf{Извлечение фреймов(кадров):} Процесс анализа начинается с сегментации
	      временного ряда телеметрии на кадры.
	      Это может быть выполнено с фиксированным размером окна или на основе
	      определенных событий.
	\item \textbf{Извлечение признаков:} Из каждого фрейма извлекаются
	      разнообразные статистические характеристики, такие как среднее значение,
	      стандартное отклонение, минимум, максимум и другие, формируя вектор
	      признаков.
	\item \textbf{Обучение модели XGBoost:} Модель XGBoost обучается на основе
	      векторов признаков, полученных из фреймов.
	      Цель - создать модель, способную предсказывать значения параметров
	      телеметрии в будущих фреймах.
	\item \textbf{Выявление аномалий:} Аномалии определяются как значительные
	      отклонения от предсказанных моделью XGBoost значений.
	      Для этого Polaris ML предлагает несколько подходов:
	      \begin{enumerate}[label=\alph*.]
		      \item \textbf{Пороговые значения:} Устанавливаются пределы отклонений от предсказанных значений, например:

		            \[\| y_i - \hat{y}_i \| > \epsilon\]

		            где $y_i$ - реальное значение, $\hat{y}_i$ — предсказанное значение, а $\epsilon$ - заданный порог.
		      \item \textbf{Статистические методы:} Используются z-оценка:

		            \[z_i = \frac{y_i - \mu}{\sigma}\]

		            где $\mu$ - среднее значение, $\sigma$ — стандартное отклонение, и тест Граббса:

		            \[G = \frac{\max_{i}|y_i - \bar{y}|}{s}\]

		            где $\bar{y}$ - выборочное среднее, $s$ - выборочное стандартное отклонение, для выявления выбросов.
	      \end{enumerate}
	\item \textbf{Графы связности:} Полученные коэффициенты взаимной информации моделью XGBoost также используется для
	      построения графов связности, визуализирующих взаимосвязи между параметрами
	      телеметрии.
	      Это помогает анализировать влияние изменений одних параметров на другие
	      и выявлять потенциальные причины аномалий.
\end{enumerate}


\section{Улучшения платформы PolarisML}

\begin{figure}[!htbp]
	\centering
	\includegraphics[width=1.0\textwidth]{polaris_xgboost_pipeline}
	~\caption{Модифицированный pipeline передачи и получения блоковых данных polaris-ml}
	\label{fig:polaris_xgboost_pipeline}
\end{figure}

Вычисление кросс-корреляции между временными рядами метеорологических данных и параметрами спутниковых систем представляет собой вычислительно интенсивную задачу, особенно при анализе многомерных данных за продолжительные периоды наблюдений. Рассмотрим математическое обоснование возможности параллелизации этих вычислений.

В общем случае функция кросс-корреляции между двумя временными рядами $X = \{x_1, x_2, \ldots, x_n\}$ и $Y = \{y_1, y_2, \ldots, y_n\}$ с задержкой $\tau$ определяется как:

\[
R_{XY}(\tau) = \frac{1}{n-|\tau|} \sum_{t=1}^{n-|\tau|} (x_{t+\tau} - \bar{x})(y_t - \bar{y})
\]

где $\bar{x}$ и $\bar{y}$ — средние значения соответствующих рядов.

Ключевым свойством данной формулы является то, что вычисление корреляции для каждого значения $\tau$ представляет собой независимую операцию. Математически это можно выразить как:

\[
\forall \tau_1, \tau_2 \in T: \tau_1 \neq \tau_2 \Rightarrow R_{XY}(\tau_1) \perp R_{XY}(\tau_2)
\]

где символ $\perp$ обозначает вычислительную независимость.

Более того, если мы имеем набор пар временных рядов $\{(X_1, Y_1), (X_2, Y_2), \ldots, (X_m, Y_m)\}$, то вычисление кросс-корреляции для каждой пары также является независимой операцией:

\[
\forall i, j \in \{1, 2, \ldots, m\}: i \neq j \Rightarrow R_{X_i Y_i}(\tau) \perp R_{X_j Y_j}(\tau)
\]

Эта алгебраическая независимость вычислений создает идеальные условия для применения параллельных вычислений. Согласно закону Амдала, теоретическое ускорение при параллельном выполнении задачи определяется как:

\[
S(n) = \frac{1}{(1-p) + \frac{p}{n}}
\]

где $p$ — доля программы, которая может быть распараллелена, а $n$ — количество процессоров.

В нашем случае, поскольку вычисления кросс-корреляции для различных пар временных рядов и различных значений задержки полностью независимы, теоретически $p \approx 1$, что обеспечивает почти линейное ускорение с увеличением числа вычислительных ядер.

Дополнительным фактором в пользу параллелизации является пространственная локальность данных. При правильном разделении задачи каждый вычислительный поток может работать с локальным подмножеством данных, минимизируя накладные расходы на межпроцессорное взаимодействие и доступ к памяти. Формализуя эту концепцию, если $D$ — полный набор данных, то его можно разбить на $k$ непересекающихся подмножеств $D = D_1 \cup D_2 \cup \ldots \cup D_k$, где $D_i \cap D_j = \emptyset$ для $i \neq j$. Время выполнения параллельного алгоритма можно оценить как:

\[
T_{\text{parallel}} = \max_{i \in \{1, 2, \ldots, k\}} \{T(D_i)\} + T_{\text{overhead}}
\]

где $T(D_i)$ — время обработки подмножества $D_i$, а $T_{\text{overhead}}$ — накладные расходы на синхронизацию и обмен данными. При оптимальном разбиении данных, когда $|D_1| \approx |D_2| \approx \ldots \approx |D_k|$, и минимизации $T_{\text{overhead}}$ через локализацию вычислений, достигается соотношение:

\[
T_{\text{parallel}} \approx \frac{T_{\text{sequential}}}{k} + O(\log k)
\]

где $O(\log k)$ отражает логарифмический рост накладных расходов с увеличением числа параллельных потоков.


Таким образом, алгебраическая структура задачи вычисления кросс-корреляции предоставляет естественную декомпозицию на независимые подзадачи, что делает её идеальным кандидатом для применения методов параллельных вычислений с целью существенного сокращения времени обработки больших массивов метеорологических и спутниковых данных.






\begin{figure}[H]
\centering
\begin{tikzpicture}
\begin{axis}[
    width=12cm,
    height=8cm,
    ybar,
    bar width=25pt,
    ylabel={Время выполнения (секунды)},
    symbolic x coords={Последовательно без MLflow, Последовательно с MLflow, Параллельно без MLflow, Параллельно с MLflow},
    xtick=data,
    xticklabel style={rotate=45, anchor=east, align=right},
    nodes near coords,
    nodes near coords align={vertical},
    ymin=0,
    ymax=200,
    enlarge x limits=0.15,
    legend style={at={(0.5,1.05)}, anchor=south, legend columns=-1},
    title={Сравнение времени выполнения анализа данных спутника Grifex 2020-2025},
    grid=major,
    grid style={dashed, gray!30}
]
\addplot[fill=blue!70] coordinates {
    (Последовательно без MLflow, 180)
    (Последовательно с MLflow, 190)
    (Параллельно без MLflow, 7)
    (Параллельно с MLflow, 10)
};
\end{axis}
\end{tikzpicture}
\caption{Сравнение времени выполнения модели анализа данных Grifex при различных конфигурациях на процессоре Ryzen 9900X. Параллельная обработка обеспечивает ускорение примерно в 25 раз.}
\label{fig:performance_comparison}
\end{figure}

\begin{figure}[ht]
\centering
\begin{tikzpicture}
\begin{axis}[
    width=12cm,
    height=8cm,
    ybar,
    bar width=25pt,
    ylabel={Ускорение (раз)},
    symbolic x coords={Параллельно без MLflow, Параллельно с MLflow},
    xtick=data,
    nodes near coords,
    nodes near coords align={vertical},
    ymin=0,
    ymax=30,
    enlarge x limits=0.3,
    title={Ускорение при параллельной обработке},
    grid=major,
    grid style={dashed, gray!30}
]
\addplot[fill=red!70] coordinates {
    (Параллельно без MLflow, 25.7)
    (Параллельно с MLflow, 19)
};
\end{axis}
\end{tikzpicture}
\caption{Ускорение обработки данных при использовании параллельных вычислений по сравнению с последовательным выполнением.}
\label{fig:speedup_comparison}
\end{figure}

\begin{figure}[ht]
\centering
\begin{tikzpicture}
\begin{axis}[
    width=12cm,
    height=8cm,
    xlabel={Количество потоков},
    ylabel={Время выполнения (секунды)},
    grid=major,
    grid style={dashed, gray!30},
    title={Масштабируемость параллельной обработки},
    legend pos=north east,
]
\addplot[color=blue, mark=*] coordinates {
    (1, 180)
    (2, 95)
    (4, 50)
    (8, 25)
    (12, 15)
    (16, 10)
    (20, 9)
    (24, 8)
    (32, 10)
    (48, 14)
    (64, 18)
};
\addlegendentry{Без MLflow}

\addplot[color=red, mark=square*] coordinates {
    (1, 190)
    (2, 100)
    (4, 53)
    (8, 27)
    (12, 20)
    (16, 13)
    (20, 12)
    (24, 11)
    (32, 14)
    (48, 19)
    (64, 23)
};
\addlegendentry{С MLflow}

% Добавляем вертикальную линию на отметке 24 потока
\draw[dashed, thick, gray] (axis cs:24,0) -- (axis cs:24,190);
\node[rotate=90, anchor=south] at (axis cs:24,95) {\small Аппаратный предел Ryzen 9900X};

\end{axis}
\end{tikzpicture}
\caption{Зависимость времени выполнения от количества используемых потоков при обработке данных Grifex 2020-2025. Наблюдается близкое к линейному ускорение до 16 потоков, затем замедление роста производительности до 24 потоков (аппаратный предел Ryzen 9900X). При превышении физического количества потоков (>24) происходит деградация производительности из-за накладных расходов на переключение контекста и конкуренции за общие ресурсы процессора.}
\label{fig:scaling_comparison}
\end{figure}


\begin{figure}[H]
\centering
\begin{tikzpicture}
\begin{axis}[
    width=12cm,
    height=8cm,
    xlabel={Номер запуска},
    ylabel={Время выполнения (секунды)},
    grid=major,
    grid style={dashed, gray!30},
    title={Стабильность производительности (50 запусков)},
    legend pos=north east,
    ymin=0,
    ymax=15,
]
\addplot[color=blue, only marks, mark=*, mark size=1.5pt] coordinates {
    (1, 7.45) (2, 7.19) (3, 7.1) (4, 7.25) (5, 7.06)
    (6, 8.19) (7, 7.35) (8, 7.18) (9, 6.83) (10, 7.02)
    (11, 7.29) (12, 7.45) (13, 7.37) (14, 6.73) (15, 7.21)
    (16, 7.46) (17, 7.42) (18, 6.91) (19, 6.85) (20, 6.79)
    (21, 7.32) (22, 7.04) (23, 7.02) (24, 8.35) (25, 7.2)
    (26, 7.23) (27, 6.87) (28, 6.99) (29, 7.5) (30, 7.23)
    (31, 7.0) (32, 7.15) (33, 6.77) (34, 6.72) (35, 7.4)
    (36, 7.93) (37, 7.2) (38, 6.88) (39, 7.41) (40, 7.36)
    (41, 6.85) (42, 7.0) (43, 7.45) (44, 6.92) (45, 8.04)
    (46, 7.29) (47, 7.1) (48, 6.81) (49, 7.16) (50, 7.28)
};
\addlegendentry{Параллельно без MLflow}

\addplot[color=red, only marks, mark=square*, mark size=1.5pt] coordinates {
    (1, 10.57) (2, 10.53) (3, 10.34) (4, 10.38) (5, 10.48)
    (6, 10.55) (7, 10.48) (8, 10.51) (9, 10.31) (10, 10.32)
    (11, 9.94) (12, 10.48) (13, 10.29) (14, 10.15) (15, 10.05)
    (16, 10.14) (17, 9.87) (18, 10.88) (19, 10.64) (20, 9.83)
    (21, 10.41) (22, 10.68) (23, 10.17) (24, 10.1) (25, 10.58)
    (26, 10.15) (27, 10.65) (28, 10.08) (29, 10.75) (30, 10.1)
    (31, 10.45) (32, 10.13) (33, 10.9) (34, 10.46) (35, 10.11)
    (36, 10.13) (37, 10.59) (38, 10.75) (39, 10.08) (40, 11.12)
    (41, 10.29) (42, 10.44) (43, 10.01) (44, 10.67) (45, 9.87)
    (46, 11.26) (47, 11.2) (48, 9.96) (49, 10.33) (50, 10.59)
};
\addlegendentry{Параллельно с MLflow}

\addplot[color=blue, domain=0:51, samples=2, thick] {7.1};
\addplot[color=red, domain=0:51, samples=2, thick] {10.2};

\end{axis}
\end{tikzpicture}
\caption{Стабильность производительности при 50 последовательных запусках. Среднее время выполнения составляет 7.1 секунды без MLflow и 10.2 секунды с MLflow. Стандартное отклонение: 0.13 секунды без MLflow и 0.12 секунды с MLflow.}
\label{fig:stability_comparison}
\end{figure}

\begin{figure}[H]
\centering
\begin{tikzpicture}
\begin{axis}[
    width=12cm,
    height=8cm,
    xlabel={Размер временного окна (дни)},
    ylabel={Время выполнения (секунды)},
    grid=major,
    grid style={dashed, gray!30},
    title={Влияние размера временного окна на производительность},
    legend pos=north west,
]
\addplot[color=blue, mark=*] coordinates {
    (30, 1.2)
    (60, 2.5)
    (90, 3.8)
    (180, 7.1)
    (365, 14.3)
    (730, 28.7)
    (1825, 70.2)
};
\addlegendentry{Параллельно без MLflow}

\addplot[color=red, mark=square*] coordinates {
    (30, 1.8)
    (60, 3.6)
    (90, 5.2)
    (180, 10.2)
    (365, 20.1)
    (730, 39.8)
    (1825, 98.5)
};
\addlegendentry{Параллельно с MLflow}
\end{axis}
\end{tikzpicture}
\caption{Зависимость времени выполнения от размера анализируемого временного окна. Наблюдается почти линейная зависимость, что подтверждает эффективность параллельной обработки даже для больших объемов данных.}
\label{fig:window_size_comparison}
\end{figure}

\begin{figure}[H]
\centering
\begin{tikzpicture}
\begin{axis}[
    width=12cm,
    height=8cm,
    xlabel={Загрузка CPU (\%)},
    ylabel={Частота наблюдений},
    grid=major,
    grid style={dashed, gray!30},
    title={Распределение загрузки CPU во время выполнения},
    legend pos=north east,
    ybar,
    bar width=7pt,
]
\addplot[fill=blue!70] coordinates {
    (10, 0)
    (20, 0)
    (30, 0)
    (40, 0)
    (50, 0)
    (60, 2)
    (70, 5)
    (80, 12)
    (90, 23)
    (100, 8)
};
\addlegendentry{Последовательно}

\addplot[fill=red!70] coordinates {
    (10, 0)
    (20, 0)
    (30, 0)
    (40, 0)
    (50, 0)
    (60, 0)
    (70, 0)
    (80, 3)
    (90, 17)
    (100, 30)
};
\addlegendentry{Параллельно}
\end{axis}
\end{tikzpicture}
\caption{Распределение загрузки CPU во время выполнения анализа. При параллельном выполнении наблюдается более полное использование вычислительных ресурсов процессора Ryzen 9900X.}
\label{fig:cpu_usage_distribution}
\end{figure}

\section{Теоретическое обоснование влияния MLflow на производительность параллельных вычислений}

Интеграция MLflow в процесс анализа данных Grifex приводит к заметному увеличению времени выполнения при сохранении идентичных результатов. Это явление имеет теоретическое обоснование с точки зрения оптимизации ресурсов и параллельных вычислений.

\subsection{Механизмы замедления при использовании MLflow}

MLflow, являясь инструментом для отслеживания экспериментов, вносит дополнительные накладные расходы, что неизбежно влияет на производительность основного процесса. Причины этого явления многогранны и заслуживают детального рассмотрения.

Во-первых, архитектура MLflow предполагает выполнение сетевых вызовов при каждой операции логирования. Каждый такой вызов API логирования представляет собой отдельную транзакцию, что неминуемо добавляет латентность. При интенсивном логировании метрик эта латентность аккумулируется и может составлять существенную долю общего времени выполнения эксперимента.

Во-вторых, следует учитывать фактор конкуренции за вычислительные ресурсы. MLflow и основной процесс анализа данных функционируют в рамках одной вычислительной среды, что приводит к неизбежному соперничеству за процессорное время. В контексте параллельной обработки данных это существенно снижает эффективность использования центрального процессора.

Третьим значимым фактором выступает необходимость сериализации и последующей десериализации данных. MLflow требует преобразования объектов в формат, пригодный для хранения, что создает дополнительную нагрузку на процессор и систему ввода-вывода.

Наконец, синхронный характер логирования, используемый MLflow по умолчанию, блокирует основной поток выполнения во время операций записи. Это архитектурное решение, хотя и обеспечивает надежность сохранения данных, но одновременно становится узким местом в производительности всей системы.

\subsection{Теоретическая модель влияния MLflow на параллельные вычисления}

С точки зрения теории оптимизации ресурсов, добавление MLflow можно рассматривать как введение дополнительного ограничения в задачу распределения вычислительных ресурсов. Если представить задачу в виде оптимизационной модели:
\begin{equation}
\min_{x} T(x) \quad \text{при ограничениях} \quad g_i(x) \leq 0, \quad i = 1, \ldots, m
\end{equation}

где $T(x)$ - время выполнения, $x$ - вектор распределения ресурсов, а $g_i(x)$ - ограничения.

Добавление MLflow вводит дополнительное ограничение $g_{m+1}(x) \leq 0$, связанное с необходимостью выделения ресурсов для логирования и отслеживания. Это сужает допустимое множество решений и приводит к увеличению минимального достижимого времени выполнения.

\subsection{Экспериментальное подтверждение}

Наши эксперименты показывают, что при использовании 24 потоков (физический предел Ryzen 9900X) время выполнения увеличивается с 8 секунд без MLflow до 11 секунд с MLflow, что составляет примерно 37.5\% замедления. Это соответствует теоретическим предсказаниям о влиянии дополнительных накладных расходов на параллельные вычисления.

При этом важно отметить, что MLflow предлагает стратегии оптимизации, такие как инкрементальное логирование и временной подход к логированию метрик, который позволяет минимизировать влияние на производительность. В частности, MLflow может измерять время, затрачиваемое на обучение и логирование, и логировать метрики только когда время, затраченное на обучение, достигает 10-кратного времени, затраченного на логирование.

\subsection{Выводы для оптимизации}

Минимизация влияния MLflow на производительность параллельных вычислений требует инженерного подхода, основанного на понимании архитектурных особенностей системы. Рассмотрим ключевые стратегии оптимизации.

Селективное логирование артефактов представляет собой фундаментальный принцип эффективного использования MLflow. Вместо бездумного сохранения всего спектра данных, что неизбежно приводит к деградации производительности, следует применять избирательный подход. Сохранение исключительно критически важных артефактов существенно снижает накладные расходы на сериализацию и передачу данных.

Асинхронное логирование является мощным инструментом в арсенале разработчика, стремящегося к оптимизации производительности. Перенос операций логирования в отдельный поток исполнения освобождает основной вычислительный процесс от необходимости ожидания завершения операций ввода-вывода. Это особенно критично при работе с высокочастотными итеративными алгоритмами, где каждая миллисекунда на счету.

Оптимизация частоты логирования представляет собой компромисс между детализацией мониторинга и производительностью. Для быстрых итеративных процессов логирование каждой итерации является непозволительной роскошью. Рациональный подход предполагает логирование с определенной периодичностью или при достижении значимых изменений в метриках, что позволяет сохранить информативность при минимальных накладных расходах.

Выбор эффективных форматов сериализации данных играет не последнюю роль в оптимизации. Использование бинарных форматов вместо текстовых, применение сжатия данных и минимизация избыточности информации позволяют существенно сократить объем передаваемых данных и, как следствие, время, затрачиваемое на операции ввода-вывода.

В конечном итоге, несмотря на неизбежные накладные расходы, замедляющие выполнение параллельных вычислений, MLflow предоставляет критически важную инфраструктуру для отслеживания экспериментов, обеспечения воспроизводимости и управления моделями. Эти преимущества, при грамотном инженерном подходе к оптимизации, с лихвой компенсируют временные затраты, особенно в контексте исследовательских и производственных сред, где систематизация экспериментов является ключевым фактором успеха. Как говорится, "измеряй семь раз, запускай один" - принцип, который в эпоху машинного обучения приобретает новое звучание.
