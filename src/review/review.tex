\documentclass[14pt, a4paper]{bsu_review}

\begin{document}

\begin{center}
\textbf{Рецензия}
\end{center}

\begin{center}
\noindent на дипломную работу студента 4-го курса\\
факультета радиофизики и компьютерных технологий\\
специальности «Аэрокосмические радиоэлектронные и информационные системы и технологии» Глеба Евгения Михайловича на тему:\\
«Оценка влияния солнечной активности на работу бортовых систем низкоорбитальных спутников с использованием алгоритма машинного обучения XGBoost и построения графа связности»
\end{center}

Дипломная работа Глеба Е.М. посвящена разработке методологии анализа влияния космической погоды на функционирование спутниковых систем с использованием современных алгоритмов машинного обучения. Исследование направлено на решение критически важной задачи прогнозирования отказов бортовых систем малых космических аппаратов в условиях воздействия солнечной активности. В условиях экспоненциального роста количества космических аппаратов на околоземной орбите проблема влияния космической погоды на функционирование спутниковых систем приобретает особую актуальность. Тема дипломной работы Глеба Е.М. является высоко актуальной.

В первой главе описывается глобальная сеть наземных спутниковых станций SatNOGS с открытым исходным кодом, рассматриваются компоненты системы и особенности работы с данными телеметрии. Во второй главе представлена разработанная платформа Polaris ML с модифицированным алгоритмом XGBoost, включающая оптимизацию гиперпараметров, параллелизацию вычислений и интеграцию MLflow для управления экспериментами. В третьей главе проведен анализ графов кросс-корреляций для пяти спутников (GRIFEX, ENSO, Veronika, LASARsat, INSPIRESat-1), выявлены закономерности влияния солнечной активности на различные типы космических аппаратов.

В рамках дипломного проекта была создана усовершенствованная система анализа телеметрических данных с применением модифицированного алгоритма XGBoost, демонстрирующая высокие показатели классификации: точность = 0.92, полнота = 0.96. Реализована параллельная обработка данных с 25-кратным ускорением (сокращение времени обработки со 120 до 4.8 секунды). Проанализированы 847,326 телеметрических записей пяти спутников за период 2021–2025 гг. Впервые применен модифицированный XGBoost с регуляризацией L1/L2 для прогнозирования отказов спутниковых систем. Предложен математический аппарат построения графов связности космической погоды с метрикой центральности по собственным векторам. Выявлены различия в реакции на солнечную активность между спутниками с радиационно-стойкими компонентами и коммерческими решениями.

В ходе написания дипломной работы Глеба Е.М. продемонстрировал глубокие навыки самостоятельного научного исследования, способность к комплексному анализу больших данных, владение современными методами машинного обучения и их адаптацией к специфике космических задач. Студент показал умение логично и последовательно излагать сложный технический материал, формулировать обоснованные выводы на основе экспериментальных данных.

Считаю, что дипломная работа заслуживает оценки «10 (десять)», а Глеба Е.М. присвоения квалификации «Специалист по аэрокосмическим радиоэлектронным и информационным системам и технологиям. Радиофизик».

\vspace{1cm}

\noindent Рецензент\\
кандидат физ.-мат. наук, доцент, \\
заведующий кафедрой интеллектуальных систем \hfill Козлова Е.И.

\end{document}
