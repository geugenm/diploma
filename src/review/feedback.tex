\documentclass[14pt, a4paper]{bsu_review}

\begin{document}

\begin{center}
\textbf{Отзыв}
\end{center}

\begin{center}
\noindent на дипломную работу студента 4-го курса\\
факультета радиофизики и компьютерных технологий\\
специальности «Аэрокосмические радиоэлектронные и информационные системы и технологии» Глеба Евгения Михайловича на тему:\\
«Оценка влияния солнечной активности на работу бортовых систем низкоорбитальных спутников с использованием алгоритма машинного обучения XGBoost и построения графа связности»
\end{center}

Дипломная работа Глеба Е.М. посвящена разработке инновационной методологии анализа воздействия космической погоды на функционирование бортовых систем современных низкоорбитальных спутников. Актуальность исследования обусловлена экспоненциальным ростом количества космических аппаратов на орбите и необходимостью повышения их надежности в условиях солнечной активности. Комплексный подход, сочетающий методы машинного обучения и теорию графов, представляет значительный научный и практический интерес для космической отрасли.

В первой главе проведен детальный анализ архитектуры глобальной сети наземных станций SatNOGS, рассмотрены методы сбора и предобработки телеметрических данных. Особое внимание уделено разработке специализированного веб-скрапера для обхода ограничений Grafana Dashboard. Во второй главе представлена модифицированная платформа Polaris ML с усовершенствованным алгоритмом XGBoost, обеспечивающим 25-кратное ускорение обработки данных. Третья глава содержит комплексный анализ графов кросс-корреляций для пяти спутников различной архитектуры, выявляющий ключевые закономерности влияния солнечной активности.

Ключевые результаты работы включают:
Создание программного комплекса с параллельной обработкой данных на GPU,
достижение точности прогнозирования отказов 0.92 при полноте 0.96, анализ 847,326 телеметрических записей за 2021-2025 гг., реализацию механизма автоматического построения графов связности, интеграцию с MLflow для управления экспериментами

Научная новизна подтверждается:
первым применением модифицированного XGBoost с L1/L2 регуляризацией, разработкой математического аппарата анализа графов космической погоды, введением метрики центральности по собственным векторам

Результаты исследования докладывались на международных конференциях и опубликованы в 4 сборниках международных конференций и симпозиумов (CDATA2024, 17-я Международная научно-техническая конференция молодых ученых и студентов в Минске, XII Международный симпозиум «Современные проблемы радиофизики», VIII Междунар. науч.-практ. конференции). Проверка в системе «Антиплагиат» показала оригинальность 97.02\%.

Считаю, что дипломная работа заслуживает оценки «10 (десять)», а Глеба Е.М. присвоения квалификации «Специалист по аэрокосмическим радиоэлектронным и информационным системам и технологиям. Радиофизик».

\vspace{1cm}

\noindent Руководитель дипломной работы\\
старший преподаватель\\
кафедры физики и аэрокосмических технологий
 \hfill Баранова В.С.

\end{document}
