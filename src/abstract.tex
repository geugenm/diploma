\chapter*{РЕФЕРАТ}

\textbf{Структура и объем}: \maincontentpages~стр., \totalfigures~рис.,
\totaltables~табл., \thetotalsources~источников.

\textbf{Ключевые слова}: КОСМИЧЕСКАЯ ПОГОДА, СОЛНЕЧНАЯ АКТИВНОСТЬ, МАШИННОЕ
ОБУЧЕНИЕ, XGBOOST, СПУТНИКОВЫЕ СИСТЕМЫ, ТЕЛЕМЕТРИЯ, ГРАФ СВЯЗНОСТИ,
НИЗКООРБИТАЛЬНЫЕ СПУТНИКИ, SATNOGS.

\textbf{Объект исследования}: низкоорбитальные спутники и их бортовые системы в
условиях воздействия космической погоды.

\textbf{Предмет исследования}: корреляции между параметрами солнечной активности
и электротехническими характеристиками космических аппаратов.

\textbf{Цель работы}: разработка методологии анализа влияния космической погоды
на функционирование спутниковых систем с использованием алгоритмов машинного
обучения.

\textbf{Методы исследования}: модифицированный \textsc{XGBoost} с
гиперпараметрической оптимизацией, анализ графов кросс-корреляций, параллельные
вычисления на \textsc{GPU}, статистический анализ временных рядов,
автоматизированный сбор телеметрии \textsc{SatNOGS}.

\textbf{Основные результаты}: Создана платформа \textsc{Polaris ML} с 
модифицированным алгоритмом XGBoost, демонстрирующая высокие показатели 
классификации: точность~$=~0.92$, полнота~$=~0.96$. 
Реализована параллельная обработка с $25{\times}$ ускорением 
(сокращение времени обработки со $120$ до $4.8$ сек.). Проанализированы 
$847{\,}326$ телеметрических записей $5$ спутников (\textsc{Grifex}, 
\textsc{Enso}, \textsc{Veronika}, \textsc{LASARsat}, \textsc{INSPIRESat-1}) 
за период $2021$--$2025$ гг.


\textbf{Научная новизна}: Впервые применен модифицированный \textsc{XGBoost} с
регуляризацией L1/L2 для прогнозирования отказов спутниковых систем. Предложен
математический аппарат построения графов связности космической погоды с метрикой
центральности по собственным векторам.

\textbf{Практическая значимость}: Сокращение времени обработки телеметрии,
повышение точности прогнозирования отказов.

\textbf{Область применения}: космические агентства, операторы спутниковых
группировок, центры прогнозирования космической погоды.




\chapter*{ABSTRACT}

\textbf{Structure and volume}: \maincontentpages~pp., \totalfigures~figs.,
\totaltables~tables, \thetotalsources~sources.

\textbf{Keywords}: SPACE WEATHER, SOLAR ACTIVITY, MACHINE LEARNING, XGBOOST,
SATELLITE SYSTEMS, TELEMETRY, CONNECTIVITY GRAPH, LOW-EARTH ORBIT SATELLITES,
SATNOGS.

\textbf{Research object}: Low-Earth orbit satellites and their onboard systems under the impact of space weather.

\textbf{Research subject}: Correlations between solar activity parameters and electrical characteristics of spacecraft.

\textbf{Research goal}: Development of a methodology for analyzing the impact of space weather on satellite system operation using machine learning algorithms.

\textbf{Research methods}: Modified \textsc{XGBoost} with hyperparameter optimization, cross-correlation graph analysis, parallel computing on \textsc{GPU}, statistical analysis of time series, automated telemetry collection via \textsc{SatNOGS}.

\textbf{Main results}: Created \textsc{Polaris ML} platform with modified XGBoost, achieving 
$\text{Precision} = 0.92$, $\text{Recall} = 0.96$ with $25{\times}$ processing speedup (from $120$ to $4.8$ seconds). Analyzed $847{\,}326$ telemetry records from $5$ satellites (\textsc{Grifex}, \textsc{Enso}, \textsc{Veronika}, \textsc{LASARsat}, \textsc{INSPIRESat-1}) over the period $2021$--$2025$.

\textbf{Scientific novelty}: Modified \textsc{XGBoost} with L1/L2 regularization applied for the first time to predict satellite system failures. Proposed mathematical apparatus for constructing space weather connectivity graphs with eigenvector centrality metrics.

\textbf{Practical significance}: Reduced telemetry processing time, improved failure prediction accuracy.

\textbf{Application area}: Space agencies, satellite constellation operators, space weather forecasting centers.





\chapter*{РЭФЕРАТ}

\textbf{Структура і аб’ём}: \maincontentpages~с., \totalfigures~рыс.,
\totaltables~табл., \thetotalsources~крын.

\textbf{Ключавыя словы}: КАСМІЧНАЕ НАДДЗЕННЕ, СОНЕЧНАЯ АКТЫЎНАСЦЬ, МАШЫННАЕ ВЫВУЧЭННЕ, XGBOOST, СУПУТНІКАВЫЯ СІСТЭМЫ, ТЭЛЕМЕТРЫЯ, ГРАФ СУВЯЗІ, НІЗКАОРБІТАЛЬНЫЯ СУПУТНІКІ, SATNOGS.

\textbf{Аб’ект даследавання}: нізкаарбітальныя супутнікі і іх бортавыя сістэмы ў умовах уздзеяння касмічнага наддзення.

\textbf{Прадмет даследавання}: карэляцыі паміж параметрамі сонечнай актыўнасці і электратэхнічнымі характарыстыкамі касмічных апаратаў.

\textbf{Мэта працы}: распрацоўка метадалогіі аналізу ўздзеяння касмічнага наддзення на функцыянаванне супутнікавых сістэм з выкарыстаннем алгарытмаў машыннага навучання.

\textbf{Метады даследавання}: мадыфікаваны \textsc{XGBoost} з гіперпараметрычнай аптымізацыяй, аналіз графаў крос-карэляцыі, паралельныя вылічэнні на \textsc{GPU}, статыстычны аналіз часавых шэрагаў, аўтаматызаваны збор тэлеметрыі \textsc{SatNOGS}.

\textbf{Асноўныя вынікі}: Створана платформа \textsc{Polaris ML} з мадыфікаваным XGBoost, якая дасягае 
$\text{Precision} = 0.92$, $\text{Recall} = 0.96$ пры $25{\times}$ паскарэнні апрацоўкі (з $120$ да $4.8$ сек.). Прааналізавана $847{\,}326$ запісаў тэлеметрыі $5$ супутнікаў (\textsc{Grifex}, \textsc{Enso}, \textsc{Veronika}, \textsc{LASARsat}, \textsc{INSPIRESat-1}) за перыяд $2021$--$2025$ гг.

\textbf{Навуковая навізна}: Упершыню выкарыстаны мадыфікаваны \textsc{XGBoost} з рэгулярызацыяй L1/L2 для прагназавання адмоваў супутнікавых сістэм. Прапанаваны матэматычны апарат пабудовы графаў сувязі касмічнага наддзення з метрыкай цэнтральнасці па ўласных вектарах.

\textbf{Практычная значнасць}: Скарачэнне часу апрацоўкі тэлеметрыі, павышэнне дакладнасці прагназавання адмоваў.

\textbf{Вобласць прымянення}: касмічныя агенцтвы, аператары супутнікавых групоў, цэнтры прагназавання касмічнага наддзення.
