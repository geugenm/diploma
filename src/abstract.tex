\chapter*{РЕФЕРАТ}

\textbf{Структура и объем}: \maincontentpages~стр., \totalfigures~рис.,
\totaltables~табл., \total{appendixsections}~прил., \thetotalsources~источников.

\textbf{Ключевые слова}: КОСМИЧЕСКАЯ ПОГОДА, СОЛНЕЧНАЯ АКТИВНОСТЬ, МАШИННОЕ
ОБУЧЕНИЕ, XGBOOST, СПУТНИКОВЫЕ СИСТЕМЫ, ТЕЛЕМЕТРИЯ, ГРАФ СВЯЗНОСТИ,
НИЗКООРБИТАЛЬНЫЕ СПУТНИКИ, SATNOGS.

\textbf{Объект исследования}: низкоорбитальные спутники и их бортовые системы в
условиях воздействия космической погоды.

\textbf{Предмет исследования}: корреляции между параметрами солнечной активности
и электротехническими характеристиками космических аппаратов.

\textbf{Цель работы}: разработка методологии анализа влияния космической погоды
на функционирование спутниковых систем с использованием алгоритмов машинного
обучения.

\textbf{Методы исследования}: модифицированный \textsc{XGBoost} с
гиперпараметрической оптимизацией, анализ графов кросс-корреляций, параллельные
вычисления на \textsc{GPU}, статистический анализ временных рядов,
автоматизированный сбор телеметрии \textsc{SatNOGS}.

\textbf{Основные результаты}: Создана платформа \textsc{Polaris ML} с
модифицированным XGBoost, достигающая $F_1\text{-Score} = 0.94$,
$\text{Precision} = 0.92$, $\text{Recall} = 0.96$ при $25{\times}$ ускорении
обработки (с $120$ до $4.8$ сек.). Проанализированы $847{\,}326$ телеметрических
записей $5$ спутников (\textsc{Grifex}, \textsc{Enso}, \textsc{Veronika},
\textsc{LASARsat}, \textsc{INSPIRESat-1}) за период $2021$--$2025Q2$ гг.

\textbf{Научная новизна}: Впервые применен модифицированный \textsc{XGBoost} с
регуляризацией L1/L2 для прогнозирования отказов спутниковых систем. Предложен
математический аппарат построения графов связности космической погоды с метрикой
центральности по собственным векторам.

\textbf{Практическая значимость}: Сокращение времени обработки телеметрии,
повышение точности прогнозирования отказов.

\textbf{Область применения}: космические агентства, операторы спутниковых
группировок, центры прогнозирования космической погоды.
