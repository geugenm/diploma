\chapter{Анализ графов спутников полученных при помощи polaris 2.0}
    
   Для оценки влияния космической погоды на бортовые системы и целевую аппаратуру малых космических аппаратов (МКА) в этой главе будет проведен комплексный анализ зависимости параметров телеметрии \cite{green_2017_impact} \cite{schlag_2018_numerical} \cite{boumghar_2018_enhanced}, индексов солнечной активности и геомагнитных индексов. В основе исследования лежит применение модели машинного обучения Polaris ML. Результаты работы модели будут представлены в виде двухмерного графа связности, отражающего выявленные взаимосвязи между исследуемыми параметрами.

    Для расчета и анализа графа связности будет использована обширная база данных телеметрии сети наземных станций SatNOGS. Параметры солнечной активности будут извлечены из следующих основных источников:
    
    \begin{itemize}
        \item Центр прогнозирования космической погоды (Space Weather Prediction Center, SWPC/SWO) \cite{swpc_noaa_data_souce};
        \item Центр наблюдения и анализа данных о влиянии Солнца в Брюсселе (Solar Influences Data Analysis Center, S.I.D.C.) \cite{silso_snd_data_source};
        \item Канадская радиоастрофизическая обсерватория в Пентиктоне \cite{swgc_flux_data_source}.
    \end{itemize}

Далее представлена таблица \ref{tab:detailed_solar_geo_params} с основными анализируемыми параметрами солнечной активности

    \begin{longtable}{|l|p{12cm}|}
        \caption{Подробное описание параметров солнечной и геомагнитной активности}
        \label{tab:detailed_solar_geo_params} \\
        
        \hline
        \textbf{Параметр} & \textbf{Описание} \\
        \hline
        \endfirsthead
        
        \multicolumn{2}{c}%
        {\tablename\ \thetable\ -- \textit{Продолжение}} \\
        \hline
        \textbf{Параметр} & \textbf{Описание} \\
        \hline
        \endhead
        
        \hline
        \multicolumn{2}{|r|}{\textit{Продолжение на следующей странице}} \\
        \hline
        \endfoot
        
        \hline
        \endlastfoot

        ssn & Среднемесячное число солнечных пятен — ключевой индикатор солнечной активности, получаемый Центром наблюдения и анализа данных о влиянии Солнца в Брюселе (S.I.D.C.). Солнечные пятна представляют собой области с пониженной температурой, вызванные магнитными полями, которые препятствуют конвективным процессам. Их количество варьируется в зависимости от 11-летнего цикла солнечной активности, что делает ssn важным параметром для понимания солнечного поведения и его влияния на космическую погоду.\\
        \hline
        smoothed\_ssn & Сглаженное число солнечных пятен — это усредненное значение количества солнечных пятен за определенный период, также предоставляемое S.I.D.C. Сглаживание позволяет устранить краткосрочные колебания и выявить долгосрочные тенденции в солнечной активности, что критически важно для прогноза космической погоды и оценки воздействия на Землю.\\
        \hline
        observed\_swpc\_ssn & Среднемесячное число солнечных пятен, зарегистрированное Центром прогнозирования космической погоды (SWPC/SWO). Этот параметр служит основой для оценки текущего состояния солнечной активности и ее потенциального влияния на магнитосферу Земли, что имеет значение для защиты спутников и других технологий.\\
        \hline
        smoothed\_swpc\_ssn & Сглаженное число солнечных пятен, полученное из наблюдений SWPC/SWO. Оно позволяет анализировать долгосрочные изменения в солнечной активности, что особенно полезно для научных исследований и разработки моделей предсказания космической погоды.\\
        \hline
        f10.7 & Среднемесячные значения потока радиоизлучения на длине волны 10,7 см — важный индикатор солнечной активности, измеряемый канадской радиоастрофизической обсерваторией в Пентиктоне, Британская Колумбия. Этот параметр коррелирует с количеством солнечных пятен и служит основным показателем для оценки интенсивности радиоволн, излучаемых Солнцем.\\
        \hline
        smoothed\_f10.7 & Сглаженные значения потока радиоизлучения 10,7 см, которые помогают устранить кратковременные колебания и выявить более стабильные тренды в солнечной радиации, что имеет критическое значение для исследований климатических изменений и космической погоды.\\
        \hline
        observed flux & Наблюдаемое значение солнечного излучения — это интегральная мера выбросов энергии от Солнца, полученная с помощью радиотелескопов. Это значение подвержено модуляции двумя основными факторами: уровнем солнечной активности и изменением расстояния между Землей и Солнцем, что делает его важным для понимания динамики солнечного излучения и его воздействия на земную атмосферу.\\
        \hline
        adjusted flux & Скорректированное наблюдаемое значение солнечного излучения, которое учитывает изменения расстояния между Землей и Солнцем, предоставляя более точную оценку энергии, достигающей нашей планеты. Этот параметр важен для климатических исследований и оценки воздействия солнечной активности на земную экосистему.\\
        adjusted flux & Наблюдаемое значение солнечного излучения, скорректированное на изменения расстояния между Землей и Солнцем и данное для среднего расстояния.\\
        \hline
        Fredericksburg A & Индекс магнитной активности в районе Фредериксбурга (США). Используется для мониторинга геомагнитных изменений. Представляет собой линейную шкалу, отражающую амплитуду возмущений магнитного поля Земли. Единица измерения: нанотесла (нТл). Диапазон значений: от 0 до 400 нТл. \\
        \hline
        Fredericksburg K 0-3 & Категории магнитной активности K-индекса (низкий уровень, 0-3) в Фредериксбурге. K-индекс измеряется каждые три часа и отражает локальные геомагнитные возмущения. Безразмерная величина. Соответствует возмущениям до 20 нТл. \\
        \hline
        Fredericksburg K 3-6 & Категории магнитной активности K-индекса (умеренный уровень, 3-6) в Фредериксбурге. Указывает на усиление геомагнитной активности. Безразмерная величина. Соответствует возмущениям от 20 до 120 нТл. \\
        \hline
        Fredericksburg K 6-9 & Категории магнитной активности K-индекса (высокий уровень, 6-9) в Фредериксбурге. Свидетельствует о сильных геомагнитных возмущениях. Безразмерная величина. Соответствует возмущениям от 120 до 300 нТл и выше. \\
        \hline
        fluxdate & Дата измерения солнечного радиоизлучения. Важна для отслеживания долгосрочных изменений солнечной активности. Формат: YYYY-MM-DD. \\
        \hline
        fluxjulian & Юлианская дата, используемая для астрономических вычислений. Обеспечивает непрерывную шкалу времени, удобную для расчетов. Единица измерения: дни. Точность: до 5 знаков после запятой. \\
        \hline
        fluxcarrington & Номер периода вращения Каррингтона для корреляции солнечного излучения. Используется для отслеживания солнечных явлений, связанных с вращением Солнца. Безразмерная величина. Период Каррингтона $\approx$ 27.2753 дня. \\
        \hline
        fluxobsflux & Наблюдаемый поток радиоизлучения (10.7 см) на момент измерения. Измеряется в солнечных единицах потока (с.е.п., 1 с.е.п. = \(10^{-22}\) Вт·м\(^{-2}\)·Гц\(^{-1}\)). Является важным индикатором солнечной активности. \\
        \hline
        fluxadjflux & Приведенный поток радиоизлучения (с поправкой на расстояние между Землей и Солнцем). Позволяет сравнивать данные, полученные в разные периоды года. Единица измерения: с.е.п. \\
        \hline
        fluxursi & Поток URSI, принятый стандарт в радиофизике для солнечного радиоизлучения. Обеспечивает стандартизированное измерение солнечного радиопотока. Единица измерения: с.е.п. \\
        \hline
        SNvalue (hemispheric) & Наблюдаемое число солнечных пятен по полушариям. Важный показатель солнечной активности, отражающий асимметрию активности Солнца. Безразмерная величина. \\
        \hline
        SNerror (hemispheric) & Ошибка в оценке числа солнечных пятен по полушариям. Указывает на точность измерений и возможные погрешности. Безразмерная величина. \\
        \hline
        Nb\_observations & Количество наблюдений, использованных для вычисления параметров. Важно для оценки статистической значимости данных. Целое число. \\
    \end{longtable}