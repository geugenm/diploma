\documentclass[aspectratio=43]{beamer}

% Язык и шрифты
\usepackage[main=russian]{babel}
\usepackage{fontspec}
\setmainfont{Times New Roman}
\setsansfont{Times New Roman}
\usepackage{csquotes}

\graphicspath{{./img/}}

% Таблицы
\usepackage{booktabs}

% Для галочек
\usepackage{amssymb}

% --- Титульный слайд ---
\title[Оценка влияния солнечной активности]{ОЦЕНКА ВЛИЯНИЯ СОЛНЕЧНОЙ АКТИВНОСТИ НА РАБОТУ БОРТОВЫХ СИСТЕМ НИЗКООРБИТАЛЬНЫХ СПУТНИКОВ С ИСПОЛЬЗОВАНИЕМ АЛГОРИТМА МАШИННОГО ОБУЧЕНИЯ XGBOOST И ПОСТРОЕНИЯ ГРАФА СВЯЗНОСТИ}
\author{
    Дипломная работа\\
    Выполнил: Глеба Е. М.\\
    Научный руководитель: Баранова В. С.
}
\institute{Белорусский государственный университет
Факультет радиофизики и компьютерных технологий
Кафедра физики и аэрокосмических технологий
}
\date{Минск, 2025}

\begin{document}

\begin{frame}[plain]
  \titlepage
\end{frame}

% --- Цель ---
\begin{frame}{Цель}
  Разработка методологии анализа влияния солнечной активности на функционирование низкоорбитальных спутников с использованием модифицированного алгоритма машинного обучения XGBoost и построения графов связности.
\end{frame}

% --- Задачи ---
\begin{frame}{Задачи}
  \begin{enumerate}
    \item Модифицировать алгоритм XGBoost с регуляризацией L1/L2 для прогнозирования отказов спутниковых систем
    \item Разработать математический аппарат построения графов связности космической погоды с метрикой центральности по собственным векторам
    \item Реализовать параллельную обработку телеметрических данных с GPU-ускорением для временных рядов объемом до 5 лет
    \item Провести валидацию модели на объемных телеметрических записях спутников (2021--2025 гг.)
    \item Создать платформу для автоматизированного анализа корреляций между солнечной активностью и параметрами бортовых систем
  \end{enumerate}
\end{frame}

% --- Объект и предмет исследования ---
\begin{frame}{Объект и предмет исследования}
  \begin{itemize}
    \item \textbf{Объект:} Бортовые системы низкоорбитальных спутников под воздействием космической погоды.
    \item \textbf{Предмет:} Корреляции между солнечной активностью и электротехническими параметрами спутников, выявляемые ML и графовым анализом.
  \end{itemize}
\end{frame}

% --- Актуальность ---
\begin{frame}{Актуальность}
  \begin{itemize}
    \item Экспоненциальный рост числа спутников на орбите
    \item Влияние космической погоды на бортовые системы усиливается
    \item Необходимость предсказания сбоев для обеспечения безопасности
    \item Экономические потери от отказов спутников
    \item Критическая важность для космических миссий
  \end{itemize}
  \vspace{1em}
  \textbf{Статистика:} Более 8000 активных спутников на орбите, ущерб от сбоев достигает миллиардов долларов ежегодно
\end{frame}

% --- Анализируемые параметры ---
\begin{frame}{Анализируемые параметры}
  \begin{tabular}{@{}l l@{}}
    \toprule
    \textbf{Параметр} & \textbf{Описание} \\
    \midrule
    ssn & среднемесячное число солнечных пятен (S.I.D.C.) \\
    f10.7 & поток радиоизлучения 10,7 см (Пентиктон, Канада) \\
    observed flux & наблюдаемое значение солнечного излучения \\
    adjusted flux & скорректированное значение солнечного излучения \\
    \bottomrule
  \end{tabular}
\end{frame}

% --- Динамика ключевых параметров ---
\begin{frame}{Динамика ключевых параметров (SWPC NOAA)}
  \centering
  \includegraphics[width=0.93\textwidth]{dynamics_key_params.png}
\end{frame}

% --- Ядро используемой модели ---
\begin{frame}{Ядро используемой модели}
  \centering
  \includegraphics[width=0.93\textwidth]{model_core.png}
\end{frame}

% --- Жизненный цикл модели ---
\begin{frame}{Жизненный цикл модели}
  \centering
  \includegraphics[width=0.93\textwidth]{model_lifecycle.png}
\end{frame}

% --- Workflow Polaris ML 2.0 ---
\begin{frame}{Workflow Polaris ML 2.0}
  \centering
  \includegraphics[width=0.93\textwidth]{workflow_polaris_ml2.png}
\end{frame}

% --- Сравнение точности алгоритмов ---
\begin{frame}{Сравнение точности алгоритмов}
  \centering
  \includegraphics[width=0.93\textwidth]{ml_accuracy_rus.png}
  \vspace{1em}
  \begin{itemize}
    \item XGBoost --- лидер по точности (F1-score: 0.94)
    \item CatBoost, LSTM --- 0.90--0.93
    \item Random Forest, SVM --- 0.85--0.87
  \end{itemize}
\end{frame}

% --- Достижения Polaris ML 2.0 ---
\begin{frame}{Достижения Polaris ML 2.0}
  \begin{itemize}
    \item  Модификация XGBoost: L1/L2-регуляризация, временные признаки 2-го порядка, снижение переобучения на 37\%
    \item  Графы связности: алгоритм с метрикой центральности по собственным векторам
    \item  Ускорение: $25 \times$ быстрее (с 120 до 4.8 сек. на 5-летних рядах)
    \item  Валидация: 847,326 телеметрических записей, 5 спутников (2021--2025), средняя ошибка прогноза: 2.1\%
    \item  Платформа Polaris 2.0: SaaS, API, интеграция с SatNOGS, Grafana
  \end{itemize}
\end{frame}

% --- Производительность AMD Ryzen 9900X ---
\begin{frame}{Производительность AMD Ryzen 9900X}
  \centering
  \includegraphics[width=0.93\textwidth]{amd_ryzen_performance.png}
\end{frame}

% --- Цель достигнута ---
\begin{frame}{Цель достигнута}
  Разработана методология анализа влияния солнечной активности на низкоорбитальные спутники, объединяющая модифицированный XGBoost с регуляризацией L1/L2 и графы связности.
  \vspace{1em}
  
  \textbf{Точность:} Precision $=0.92$, Recall $=0.96$.
\end{frame}

% --- Список собственных публикаций по теме ---
\begin{frame}{Список собственных публикаций по теме}
  \begin{itemize}
    \item Глеба Е.М., Баранова В.С. Методы оценки влияния космической погоды на бортовые системы низкоорбитальных спутников // CDATA2024.
    \item Глеба Е.М., Баранова В.С. и др. Методы оценки влияния космической погоды на бортовые системы активных спутников // Новые направления развития приборостроения, 2024.
    \item Глеба Е.М. Моделирование графов связности для анализа корреляционных зависимостей в телеметрии космических аппаратов // Сборник БГУ, 2023.
    \item Глеба Е.М., Саечников В.А. и др. Применение модифицированного алгоритма XGBoost для прогнозирования отказов спутниковых систем // XII МПР, 2024.
  \end{itemize}
\end{frame}

% --- Спасибо за внимание! ---
\begin{frame}[plain]
  \centering
  \vspace{2em}
  \LARGE Спасибо за внимание!
\end{frame}

\end{document}
