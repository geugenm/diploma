\titleformat{\section}[block]{\large\bfseries\filcenter}{}{0em}{}
\chapter*{Заключение}

В рамках данной работы было проведено комплексное исследование влияния солнечной
активности на функционирование малых космических аппаратов с использованием
методов машинного обучения и анализа графов кросс-корреляций. Основные
результаты исследования можно сформулировать следующим образом.

Разработана и реализована усовершенствованная методика анализа взаимосвязей
между параметрами солнечной активности и инженерными характеристиками
космических аппаратов на базе платформы Polaris ML. Ключевые улучшения включают:

\begin{enumerate}
	\item Модификацию базового алгоритма XGBoost с учетом специфики спутниковых данных, включая оптимизацию функции потерь, введение временных признаков второго порядка и реализацию каскадного подхода с поэтапным уточнением предсказаний.

	\item Расширение возможностей конфигурирования модели путем увеличения числа оптимизируемых гиперпараметров до 30, что позволило повысить точность модели с F1-Score 0.87 (Random Forest) до 0.94 (модифицированный XGBoost) при одновременном снижении времени обучения с 45 до 27 минут.

	\item Внедрение параллельных вычислений, обеспечивших ускорение обработки данных в 25 раз (с 180 до 7 секунд) для временных рядов объемом до 5 лет при использовании 24 вычислительных потоков.

	\item Интеграцию MLflow для стандартизации ведения отчетности и управления экспериментами, что значительно упростило сравнительный анализ моделей с разными гиперпараметрами ценой незначительного снижения производительности (увеличение времени обработки с 7 до 10 секунд).
\end{enumerate}

Проведенный структурный анализ графов кросс-корреляций для спутников GRIFEX,
ENSO, Veronika, LASARsat и INSPIRESat-1 выявил как общие закономерности, так и
специфические особенности их взаимодействия с солнечной активностью:

\begin{enumerate}
	\item Для спутника GRIFEX, оснащенного радиационно-стойкими материалами, высокая солнечная активность преимущественно оказывает положительное влияние на функционирование систем за счет увеличения энергетических возможностей при сохранении устойчивости к ионизирующему излучению.

	\item Для спутников с использованием коммерческих компонентов (ENSO, Veronika, LASARsat) характерна более сложная нелинейная структура взаимосвязей: умеренное увеличение солнечной активности стабилизирует работу, в то время как экстремальные события провоцируют каскадные сбои. При этом для Veronika наблюдается особая чувствительность к тепловым режимам, а для LASARsat критически важна стабильность ориентации в пространстве.

	\item INSPIRESat-1, благодаря более сложной системе терморегуляции и увеличенным габаритам (9U CubeSat), демонстрирует промежуточную устойчивость к внешним воздействиям. Анализ графов выявил сильную связь между режимами работы научной аппаратуры (DAXSS) и солнечным вектором, указывающую на адаптивность управления в зависимости от внешних условий.
\end{enumerate}

Полученные результаты имеют как теоретическую, так и практическую значимость. С
теоретической точки зрения, предложен математически обоснованный подход к
параллельному вычислению кросс-корреляций между временными рядами, что вносит
вклад в развитие методов анализа больших данных в космической отрасли. С
практической стороны, созданная система позволяет существенно сократить время
обработки телеметрии малых космических аппаратов и выявлять нетривиальные
взаимосвязи между солнечной активностью и функционированием бортовых систем.

Дальнейшее развитие работы возможно в направлении интеграции асинхронного
логирования в MLflow для минимизации влияния на производительность, расширения
набора анализируемых спутников для обеспечения статистической значимости, а
также совершенствования модели оптимизации гиперпараметров с применением
байесовских методов вместо сеточного поиска.
