\documentclass[14pt, a4paper]{src/bsu}

\usepackage{titlesec}
\usepackage{titling}
\usepackage{verbatim}
\usepackage{csquotes}
\usepackage{longtable}
\usepackage{subcaption}
\usepackage[backend=biber, style=numeric, sorting=title]{biblatex}

\DeclareSortingTemplate{title}{
  \sort{
    \field{title}
  }
}

\addbibresource{src/references.bib}
\DeclareNameAlias{sortname}{family-given}

\title{Оценка влияния солнечной активности на работу бортовых систем
  низкоорбитальных спутников с использованием алгоритма машинного обучения
XGBoost и построения графа связности}
\author{Глеба Е. М., Баранова В. С.}
\date{\today}

\begin{document}

\nonPrefixChapter{Введение}

Wow \cite{green_2017_impact}

\newpage


\newpage

% Conclusion (centered section format)
\titleformat{\section}[block]{\large\bfseries\filcenter}{}{0em}{}
\nonPrefixChapter{Заключение}

\newpage

% Bibliography
\printbibliography[heading=bibintoc,title={Список использованной литературы}]

\newpage

% Appendices
\appendix

\renewcommand{\chaptermark}[1]{\markboth{}{}}
\renewcommand{\sectionmark}[1]{\markright{\arabic{section}.\ #1}}

% Fix for malformed section formatting
\titleformat{\section}[block]{\large\bfseries\filcenter}{}{0em}{}

\nonPrefixChapter{Приложения} \label{sec:attachements}

\section{Приложение \arabic{section}}
\label{subsec:old_polaris_learn_config}

В данном приложении представлена старая конфигурация, используемая для формирования нейронного ансамбля с помощью алгоритма XGBoost.

%\lstinputlisting[language=Java, label={lst:old_polaris_config}, caption=Конфигурация XGBoost]{../code/old_polaris_cfg.json}

\end{document}
