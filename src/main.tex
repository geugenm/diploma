%----------------------------------------------------------------------------------------
% Additional Resources

% Comprehensive TeX documentation: https://www.tug.org/texlive/doc.html
% LaTeX community forums and support: https://tex.stackexchange.com/
% Package documentation and examples: Refer to the respective package websites or CTAN (https://ctan.org/).

%----------------------------------------------------------------------------------------

% Useful resource - https://detexify.kirelabs.org/classify.html

\documentclass[14pt, a4paper]{bsu}

\usepackage{titlesec} % Customize section headings
\usepackage{titling}
\usepackage{verbatim}
\usepackage{csquotes}
\usepackage{longtable}
\usepackage{subfig}
\usepackage[backend=biber, style=numeric, sorting=title]{biblatex}

\DeclareSortingTemplate{title}{
  \sort{
    \field{title}
  }
}

\hyphenation{Sat-NOGS}
\hyphenation{Xg-Boost}
\hyphenation{xg-boost}
\hyphenation{Po-la-ris}
\hyphenation{multi-purpose}
\hyphenation{Con-ference}
\hyphenation{Hu-man-kind}

\addbibresource{src/main.bib}

\DeclareNameAlias{sortname}{family-given}

% \usepackage{texdoc} install this package for IDE quick documentation

\title{Оценка влияния солнечной
активности на работу бортовых
систем низкоорбитальных
спутников с использованием
алгоритма машинного обучения
XGBoost и построения графа
связности}
\author{Глеба Е. М., Баранова В. С.}
\date{\today}

\graphicspath{{img}}

\begin{document}
    \nonPrefixChapter{Введение}

      

    \newpage


   

    \newpage

    \titleformat{\section}[block]{\large\bfseries\filcenter}{}{0em}{}
    \nonPrefixChapter{Заключение}

  

    \newpage

    \printbibliography[heading=bibintoc,title={Список использованной литературы}]

    \newpage

    \appendix

    \renewcommand{\chaptermark}[1]{\markboth{}{}}
    \renewcommand{\sectionmark}[1]{\markright{\arabic{section}.\ #1}}
    
    \titleformat{\section}[block]{\hspace{\parindent}\textnormal\bfseries}{}{0em}{}
    
    \nonPrefixChapter{Приложения} \label{sec:attachements}
    
    \section{Приложение \arabic{section}} 
    \label{subsec:old_polaris_learn_config}
    
    В данном приложении представлена старая конфигурация, используемая для формирования нейронного ансамбля с помощью алгоритма XGBoost.
    
    \lstinputlisting[language=Java, label={lst:old_polaris_config}, caption=Конфигурация XGBoost]{../listings/old_polaris_cfg.json}
    
\end{document}
